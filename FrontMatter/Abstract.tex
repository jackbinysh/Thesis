Knotted fields are physical fields containing knotted, linked, or otherwise topologically interesting structure. They occur in a wide variety of physical systems --- fluids, superfluids, electromagnetism, optics and high energy physics to name a few. Far from being passive structures, the occurrence of knotting in a physical field often modifies its overall properties, rendering their study interesting from both a theoretical and practical point of view. In this thesis, we focus on knotted fields in `soft matter' systems, systems which may be loosely characterised as those in which geometry plays a fundamental role, and which undergo substantial deformations in response to external forces, changes in temperature etc. Such systems are often experimentally accessible, making them natural testbeds for exploring knotted fields in all their guises.  

After providing an introduction to knotted fields with a focus on soft matter in the first chapter, in the second we introduce a method of explicitly constructing such fields based on Maxwell's solid angle construction. The method is independent of the particular system under study, and applies to arbitrary knot geometry and topology. We discuss its theory, emphasising a fundamental homotopy formula as unifying methods for its computation, as well as describing a naturally induced curve framing , which we show is related to the writhe of the curve before using it to characterise the local structure in the neighbourhood of the knot. We then move on to its practical implementation, giving examples of its use and providing C code. In subsequent chapters we use this methodology to initialise simulations in our study of knotted fields in two soft matter systems: excitable media and twist-bend nematics. In excitable media we provide a systematic survey of knot dynamics up to crossing number eight, finding generically unsteady behaviour driven by a wave-slapping mechanism. Nevertheless, we also find novel complex knotted structures and characterise their geometry and steady state motion, as well as greatly expanding upon previous evidence to demonstrate the ability of the dynamics to untangle geometries without reconnection. In twist-bend nematics we describe their fundamental geometry, that of bend. The zeros of bend are a set of lines with rich geometric and topological structure. We characterise their local structure, describe how they are canonically oriented and discuss a notion of their self-linking. We then describe their topological significance, showing that these zeros compute Skyrmion and Hopfion numbers, with accompanying simulations in twist-bend nematics.   

